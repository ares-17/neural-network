\chapter{Intro}
La relazione descrive le caratteristiche di un algoritmo in grado di addestrare una rete neurale di tipo \underline{feed-forward e full-connected}\footnotemark con uno strato interno ed uno esterno. \\
La fase di training della rete è compiuta sul dataset MNIST, un dataset di digit dei numeri 0-9. Deriva che la rete abbia come strato di output uno strato immutabile nella sua dimensione di dieci neuroni, uno per ogni categoria del problema da classificare. \\
In particolare la rete utilizza una versione dell'algoritmo di back propagation con gradient descent applicato con la variabile \texttt{momentum}\footnotemark.\\
Il progetto python proposto tenta di analizzare le performance della rete confrontando i diversi risultati ottenuti variando tre parametri : \texttt{learning rate}, \texttt{momentum} e numero neuroni interni, fissando al contempo le funzioni di attivazione, di errore.\\
Segue una sezione per l'analisi del progetto e una sezione di analisi delle performance ottenute confrontando i vari risultati. Infine si propongono nuovi sviluppi.

\footnotetext{La rete presentata è feed-forward poiché non presenta cicli ed è full-connected poiché ogni neuroni di uno strato $i$ è connesso ad ogni altro neurone dello strato $i-1$.}

\footnotetext{La variabile \texttt{momentum} è aggiunta all'algoritmo di aggiornamento dei pesi per velocizzare l'apprendimento in zone piatte, ovvero dove la derivata del peso ricada in una zona con valori pressoché costanti}