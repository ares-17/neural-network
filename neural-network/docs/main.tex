\documentclass[a4paper,11pt]{book}
\usepackage[T1]{fontenc}
\usepackage[utf8]{inputenc}
\usepackage[italian]{babel}
\usepackage{lmodern}
\usepackage{amsmath}
\usepackage{amsfonts}
\usepackage{amssymb}
\usepackage{amsthm}
\usepackage{graphicx}
\usepackage{color}
\usepackage{url}
\usepackage{theorem}
\usepackage{textcomp}
\usepackage{listings}
\usepackage{hyperref}
\usepackage{parskip}
\usepackage{soul} % for the command \hl
\usepackage{colortbl} % colore tabella
\usepackage{tikz} % per flowcharts
\usepackage{titlesec}
\usepackage{listings} % code snippet
% \usepackage[dvipsnames]{xcolor} da risolvere per ulteriori colori per snippet , come : NavyBlue
\usepackage{xcolor}
\usepackage{float} % with \begin{table}[H] for fixed table position
\usetikzlibrary{shapes, arrows} % per flowcharts
\graphicspath{{images/}}

\title{Neural Network and Deep Learning}
\author{Schiavo Alessandro}
\date{\today}

\titleformat{\chapter}[display]
  {\normalfont\bfseries}{}{0pt}{\Huge} % remove label "Chapter" for every new chapter

% code snipper colors
\definecolor{codegreen}{rgb}{0,0.6,0}
\definecolor{codegray}{rgb}{0.5,0.5,0.5}
\definecolor{codepurple}{rgb}{0.58,0,0.82}
\definecolor{backcolour}{rgb}{0.95,0.95,0.92}

% Definig a custom style:
\lstdefinestyle{mystyle}{
    backgroundcolor=\color{backcolour},   
    commentstyle=\color{codepurple},
    %keywordstyle=\color{NavyBlue},
    keywordstyle=\color{blue},
    numberstyle=\tiny\color{codegray},
    stringstyle=\color{codepurple},
    basicstyle=\ttfamily\footnotesize\bfseries,
    breakatwhitespace=false,         
    breaklines=true,                 
    captionpos=t,                    
    keepspaces=true,                 
    numbers=left,                    
    numbersep=5pt,                  
    showspaces=false,                
    showstringspaces=false,
    showtabs=false,                  
    tabsize=2
}
% -- Setting up the custom style:
\lstset{style=mystyle}

\newcommand{\ilmath}[1]{$\mathrm{#1}$}

% definizione colore headers tabelle
\definecolor{header}{RGB}{247,230,113}

\begin{document}

\maketitle
\tableofcontents

\chapter{Intro}
La relazione descrive le caratteristiche di un algoritmo in grado di addestrare una rete neurale di tipo \underline{feed-forward e full-connected}\footnotemark con uno strato interno ed uno esterno. \\
La fase di training della rete è compiuta sul dataset MNIST, un dataset di digit di numeri 0-9. Deriva che la rete abbia come strato di output uno strato immutabile nella sua dimensione di dieci neuroni, uno per ogni categoria del problema da classificare. \\
In particolare la rete utilizza una versione dell'algoritmo di back propagation con gradient descent ed applicato con la variabile \texttt{momentum}\footnotemark.\\
Il progetto python proposto tenta di analizzare le performance della rete confrontando i diversi risultati ottenuti e variando tre parametri : \texttt{learning rate}, \texttt{momentum} e numero neuroni interni, fissando al contempo le funzioni di attivazione e di errore.\\
Segue una sezione per l'analisi del progetto e una sezione di analisi delle performance ottenute confrontando i vari risultati. Infine si propongono nuovi sviluppi.\\
Per una maggiore chiarezza sono elencate porzioni di codice esemplificative per aiutare il lettore ad seguire concettualmente l'implementazione in esame. Le porzioni di codice fornite sono ridotte di righe e variabili per favorirne la lettura. 

\footnotetext{Una rete è \texttt{feed-forward} se non presenta cicli e \texttt{full-connected} se ogni neurone $j$ di uno strato $i$ è connesso ad ogni altro neurone dello strato $i-1$.}

\footnotetext{La variabile \texttt{momentum} è aggiunta all'algoritmo di aggiornamento dei pesi per velocizzare l'apprendimento in zone piatte, ovvero dove la derivata del peso ricada in una zona con valori pressoché costanti}
\chapter{Implementazione}
Il progetto è scritto in \href{https://www.python.org/downloads/release/python-3110/}{Python 3.11} e segue le caratteristiche introdotte, rimanendo flessibile per l'introduzione di nuove feature. La libreria magiormente utilizzata è \texttt{numpy} che favorisce la scrittura di codice meno propenso ad errori offrendo metodi in grado di ottimizzare i calcoli compiuti. \\
Di seguito una breve introduzione della disposizione e caratteristiche dei file:
\begin{itemize}
    \item \textbf{main.py} : inizializza gli oggetti principali ed esegue la fase di addestramento della rete in base ai parametri definiti, per poi valutarne i risultati
    \item \textbf{train.py} : definizione dei metodi per la fase di addestramento oltre che funzioni di attivazione e di errore
    \item package \textbf{model}:
    \begin{itemize}
        \item \textbf{Dataset.py}: carica il dataset MNIST negli insiemi train, validation, error con i corrispettivi insiemi di valori target 
        \item \textbf{Analysis.py} : definisce metodi per la creazione di grafici comparativi
        \item \textbf{Layer.py} : astrae e incapsula uno strato (interno o non che sia), fornendo i metodi per la fase di addestramento
        \item \textbf{Properties.py} : definizione di metodi per lettura dei parametri di test stabiliti.
    \end{itemize}
\end{itemize}

\subsection{Definizione parametri}
Il paragrafo della definizione dei parametri è anteposto ad altri di maggiore importanza per consentire al lettore una facile interpretazione dei paragrafi successivi. \\
I parametri sono definiti nel file \underline{properties.ini} nella seguente forma:
\begin{lstlisting}[language=C]
[main]
configuration = test
...
[test]
neurons = 10
momentum =  0
epochs = 100
learning_rate = 0.0001
act_functions = tanh, identity
error_function = cross-entropy-softmax
\end{lstlisting}
Dove il valore definito in \texttt{configuration} indica quale serie di parametri sono considerati, in tal caso quelli contenuti nel gruppo "test". \\
La variabile \texttt{neurons} indica il numero di neuroni contenuti nel singolo strato interno, mentre lo strato di output è composto in ogni modo da 10 neuroni.\\
La variabile \texttt{momentum} è un gruppo di valori, separati da virgole, che indica per ogni iterazione del test quali valori considerare. A tutti gli strati sono associati gli stessi valori del momentum considerato. \\
Il termine \texttt{epochs} indica il numero di epoche applicate a tutti i casi di test. \\
Il \texttt{learning\_rate} indica la serie di valori da considerare nell'aggiornamento dei pesi. Ogni valore è considerato sull'intera rete. \\
\texttt{act\_functions} specifica le due funzioni di attivazioni considerate rispettivamente per lo strato interno e lo strato di output. \\
\texttt{error\_function} indica quale funzione di errore utilizzare. La scelta ricade tra : \texttt{cross-entropy} , \texttt{cross-entropy-softmax}, \texttt{mean-square-error}. Per ottenere risultati consistenti l'analisi è proposta con la funzione \texttt{cross-entropy-softmax} come funzione di errore e \texttt{tanh} e \texttt{identità} come funzioni di attivazione.

\subsection{Creazione della rete}
La creazione della rete si basa sulla definizione dello strato interno e dello strato di output specificando le funzioni di attivazione, le corrispettive derivate, nonché il numero di neuroni dello strato interno.
\begin{lstlisting}[language=Python]
def get_layers(neurons, momentum, columns):
    return [
        Layer((neurons, columns), ReLU, ReLU_deriv, momentum), 
        Layer((10, neurons), softmax, ReLU_deriv, momentum)
    ]
\end{lstlisting}
Il parametro \texttt{neurons} contiene il numero di neuroni definito nel file \underline{properties.ini}, mentre \texttt{columns} il numero di feature dello strato di input che nel dataset MNIST corrisponde a 28x28 celle di una singola immagine. \\
L'istanziazione della classe \texttt{Layer} avviene con la chiamata al costruttore:
\begin{lstlisting}[language=Python]
class Layer:
    def __init__(self, shape, activation, derivative, momentum=0):
        self.W = np.random.normal(0, 0.1, (shape[0], shape[1]))
        self.B = np.random.normal(0, 0.1, (shape[0], 1))
        self.activation = activation
        self.derivative = derivative
        self.momentum = momentum
        self.dW_prev = np.zeros_like(self.W)
        self.db_prev = np.zeros_like(self.B)
        self.A, self.Z, self.dZ, self.db, self.dW = None, None, None, None, None
\end{lstlisting}
Le matrici di pesi e bias sono definite con valori generati casualmente seguendo una distribuzione uniforme gaussiana. I membri \texttt{dW\_prev} e \texttt{db\_prev} definiscono la derivata dei pesi precedente e la derivata dei bias precedente ed hanno le stesse dimensioni matriciali delle rispettive matrici. Essi risultano utili nel calcolo dell'aggiornamento dei pesi applicando il momentum. \\

\subsection{Fase di Training}
La fase di training è definita dal metodo principale \texttt{train} e definisce i passaggi base dell'algoritmo.
\begin{lstlisting}[language=Python]
def train(ds,layers,alpha,iterations,error_function):
    accuracy, error_train, error_valid = #... empty arrays
    for i in range(iterations):
        forward_prop(ds.train_data, layers)
        backward_prop(ds.train_data, ds.train_label ..)
        update_params(alpha, layers)

        accuracy[i] = current_accuracy(ds.test...)
        error_train[i] = get_error(ds.train ...)
        error_valid[i] = get_error(ds.valid ... )

    return error_train, error_valid, accuracy.max()
\end{lstlisting}
Per ogni iterazione è compiuta una predizione per ogni input del dataset chiamando il metodo \texttt{forward\_prop}, per poi eseguire l'algoritmo di back propagation determinando l'errore compiuto. Dall'errore compiuto sono calibrati valori dei pesi e dei basi con il metodo \texttt{update\_params}. \\
Il metodo di propagazione in avanti è definito come :
\begin{lstlisting}[language=Python]
def forward_prop(X, layers):
    input_layer = X
    for layer in layers:
        layer.forward_prop(input_layer)
        input_layer = layer.A

class Layer:
    # ... other functions
    def forward_prop(self, input):
        self.Z = self.W.dot(input) + self.B
        self.A = self.activation(self.Z)
\end{lstlisting}
Nel quale ad ogni passo è ridefinita la variabile \texttt{input\_layer} come output dello strato precedente, assegnando allo strato successivo i valori del dataset. \\
L'algoritmo di back propagation è definito come:
\begin{lstlisting}[language=Python]
def backward_prop(X, one_hot_Y, layers, error_deriv):
    input_layers = [X]
    for index in range(len(layers) - 1):
        input_layers.append(layers[index].Z)
    
    dZ = error_deriv(layers[-1].Z , one_hot_Y) *
        layers[-1].derivative(layers[-1].A)
    for index in range(len(layers) - 1, -1, -1):
        current = layers[index]
        current.backward_prop(dZ, input_layers[index])
        if index - 1 > - 1:
            dZ = current.W.T.dot(dZ) *
                layers[index-1].derivative(layers[index-1].A)

class Layer:
    # ... other functions
    def backward_prop(self, dZ, input, m):
        self.dZ = dZ
        self.dW = self.dZ.dot(input.T)
        self.db = np.sum(self.dZ)
\end{lstlisting}
La funzione raccoglie prima i valori di output di ogni strato nel vettore \texttt{input\_layers} per poi calcolare i valori delta $\delta$ e le corrispettive derivate delle funzioni di attivazione sui valori di output determinati. Il ciclo sfrutta di passi i iterazione per determinare il delta dello strato successivo iterando la rete dall'ultimo al primo strato. \\
La definizione delle funzioni si basa sul calcolo ricorsivo del delta per lo strato corrente.
Il delta è calcolato secondo due regole che discriminano la tipologia di strato sul quale è calcolato; da qui in poi $\delta_L$ denota il valore calcolato sull'ultimo strato, $\delta_h$ denota il valore di uno dei qualsiasi strati interni.
\begin{align*}
\delta_L = g^{\prime}_L(A^L) * \frac{\partial E}{\partial y_k}
\end{align*}
Nella formula il primo termine indica la derivata della funzione di attivazione dello strato di output , il secondo indica la derivata della funzione di errore rispetto al valore di uscita prodotto dallo strato. \\
La seguente formula descrive il calcolo del delta per gli strati interni:
\begin{align*}
\delta_i^l = g^{\prime}_h(a^l_i) * \sum^{ml+1}_{j=1}{W_{j,i}^{l+1} \delta_j^{l+1}}
\end{align*}
I pedici $i$ e $j$ indicano i neuroni considerati. Al contempo considerando $A^l$ come la matrice degli input, $W^l$ matrice dei pesi e $\delta^l$ i delta dei neuroni dello strato corrente, allora :
\begin{align*}
\delta^l = g^{\prime}_h(A^l) * (W^{l+1})^T \delta^{l+1}
\end{align*}
L'algoritmo della back propagation implementato sfrutta proprio il concetto di matrici: i dati sono raggruppati in matrici facilitando la scrittura del codice e ottenendo una maggiore efficienza in combinazione con un aggiornamento di tipo batch.\\
L'algoritmo di aggiornamento dei pesi si basa su un aggiornamento di tipo \underline{batch}: tutti i pesi della rete sono aggiornati simultaneamente dopo aver derivato l'errore compiuto durante la singola iterazione d'apprendimento.\\
L'algoritmo di aggiornamento è composto da:
\begin{lstlisting}[language=Python]
def update_params(alpha, layers):
    for layer in layers:
        layer.update_params(alpha)

class Layer:
    # ... other functions
    def update_params(self, alpha):
        self.dW = self.momentum * self.dW_prev
            - alpha * self.dW
        self.db = self.momentum * self.db_prev 
            - alpha * self.db
        self.W += self.dW
        self.B += self.db
\end{lstlisting}
La regola applicata è la seguente:
\begin{align*}
w_{i,j} = w_{i,j} (- \eta * \frac{d}{dw_{i,j}}E^t + \alpha \cdot \Delta w_{i,j}^{t-1})
\end{align*}
La regola di aggiornamento della discesa del gradiente, applicata considerando il parametro \texttt{momento}, considera il precedente valore assunto dal parametro con un tasso variabile indicato proprio dal momento $\alpha$. Il parametro risulta particolarmente utile per superare regioni di plateau, dove il valore e la direzione del peso rimangono invariati.  \\
La regola della discesa del gradiente deriva il nuovo valore assoluto dalla moltiplicazione tra \texttt{learing\_rate} e derivata parziale della funzione di errore rispetto al peso corrente $w_{i,j}$. Un learning rate alto può portare una rapida convergenza del modello, rischiando di saltare un eventuale minimo globale o di oscillare intorno ad esso. Al contempo un valore basso per un caso specifico può causare aggiornamenti più piccoli, rallentando la convergenza. Allo stesso modo è di cruciale importanza porre un giusto valore al parametro $\alpha$.

\subsection{Funzioni applicate}
Di seguito sono presentate le funzioni applicate durante la fase di training e di evaluation dei risultati ottenuti, correlate da definizione matematica. Le funzioni denotate con $'$ sono le corrispondenti derivate parziali della funzione in esame.
\subsubsection{Funzione softmax}
La funzione softmax è una funzione applicata all'output della rete con lo scopo di \underline{normalizzare i risultati} ottenuti durante il processo di training. Il processo di normalizzazione applicato dalla funzione deriva da valori grezzi ricavati dai nodi di output di una rete, applicata su un problema di classificazione, e produce un vettore di probabilità che indica quanto l'input in esame appartenga alla classe considerata. Formula matematica:
\begin{align*}
\text{S}(z_i) = \frac{e^{z_i}}{\sum_{j=1}^{N} e^{z_j}}
\end{align*}
L'input $z_i$ della funzione è l'output di un nodo dell'ultimo strato e tale valore utilizzato come esponente di $e$ è diviso dalla somma di tutti i valori di output anch'essi posti come esponente della costante $e$.
\subsubsection{Funzione tanh}
La funzione tanh calcola la tangente iperbolica del valore di input. La tangente iperbolica è una funzione che mappa i valori in un intervallo compreso tra -1 e 1 ed è simmetrica rispetto all'origine.\\
Formule:
Formula matematica:
\begin{align*}
\text{tanh}(x) = \frac{e^x - e^{-x}}{e^x + e^{-x}} 
\end{align*}
\begin{align*}
\text{tanh}^{\prime}(x) = 1 - tanh(x)^2
\end{align*}
\subsubsection{Funzione sigmoide}
La funzione sigmoide traforma il valore di input in un valore compreso tra 0 ed 1. Per valori sempre più elevati il valore di ritorno dela funzione tende ad 1, in maniera speculare il valore tende a 0.
\begin{align*}
sigmoid(x) = \frac{1}{(1 + exp(-x))}
\end{align*}
\begin{align*}
sigmoid^{\prime}(x) = sigmoid(x)(1 - sigmoid(x))
\end{align*}
\subsubsection{Funzione di errore cross-entropy}
La funzione cross entropy è una funzione che misura la discrepanza tra i valori predetti e valori target di un apprendimento di tipo supervisionato ed è applicata nei problemi di classificazione.
\begin{align*}
CE(p,q) = -\sum_{i} p(i) \log(q(i))
\end{align*}
\begin{align*}
CE(p,q)^{\prime}= -\frac{p(i)}{q(i)}
\end{align*}
La funzione compara i valori di due distribuzioni di probabilità e nei casi in cui la funzione di attivazione dell'ultimo strato di una rete neurale non applichi una formula di normalizzazione dei valori rispetto alle probabilità di appartenenza nelle classi, risulta necessario ridefinire la funzione di errore includendo tale fase con ad esempio il softmax.\\
La regola della funzione cross entropy applicata con softmax è :
\begin{align*}
CE(p,q) = -\sum_{i} p(i) \log(q(i))
\end{align*}
\begin{align*}
CE(p,q)^{\prime} = -\frac{p(i)}{q(i)}
\end{align*}
\subsubsection{Funzione sum of squares}
La funzione sum of squares calcola attraverso l'elevamento a potenza e della sottrazione, la differenza tra due distribuzioni di probabilità ed è comunemente utilizzata nei problemi di regressione.
\begin{align*}
\text{s}(x) = \sum_{i=1}^{n} (y_i - t_i)^2
\end{align*}
\begin{align*}
\frac{\partial \text{s}(x)}{\partial x_i} = 2 (y_i - t_i)
\end{align*}

\subsection{Calcolo accuratezza}
Il calcolo dell'accuratezza è compiuto eseguendo la divisione tra il numero di elementi correttamente predetti e il numero totale di input del caso di test.
\begin{align*}
\text{accuracy} = \frac{\text{\# label corrette}}{\text{\# totale label}}
\end{align*}

\subsection{Fase di Analisi}
I dati raccolti durante la fase di addestramento sono poi interpretati con grafici nella fase di analisi. \\
La fase di analisi è compiuta dalla classe \texttt{Analysis} che ad ogni esecuzione di nuovo addestramento e con una nuova combinazione dei parametri definiti dall'utente, raccoglie: l'errore compiuto sul training e validation set, oltre che l'accuratezza ottenuta, ad ogni epoca, sul test set. \\
\begin{lstlisting}[language=Python]
class Analysis:
    def __init__(self):
        self.accuracies = []
        self.errors_train = []
        self.errors_valid = []
        self.test_accuracy = []
        self.results = {}

    def partial(self,neurons,rate,momentum,error_train,error_valid,accuracy):
        if neurons not in self.results:
            self.results[neurons] = {}
        if rate not in self.results[neurons]:
            self.results[neurons][rate] = {}
        self.results[neurons][rate][momentum] = {
            'error_train' : error_train,
            'error_valid' : error_valid,
            'accuracy' : accuracy
        }

    def save_charts(self):
        for neurons, nested_dict in self.results.items():
            for learning_rate, nested_nested_dict in nested_dict.items():
                for momentum, metriche in nested_nested_dict.items():
                    error_train = metriche['error_train']
                    error_valid = metriche['error_valid']
                    accuracy = metriche['accuracy']

                    plt.plot(error_train, label='Train Error')
                    plt.plot(error_valid, label='Valid Error')
                    plt.xlabel('Epoch')
                    plt.ylabel('Error')
                    plt.title(...)
                    plt.legend()
                    plt.savefig(self.get_result_path_error(...))
                    plt.close()

    def get_result_path_error(self, name):
        return os.path.join(os.getcwd(), "results/errors", name + ".png")
\end{lstlisting}
Il metodo \texttt{partial} raccoglie i risultati ottenuti ad ogni addestramento, mentre il metodo \texttt{save\_charts()} salva i risultati in un grafico che compara l'andamento dell'errore compiuto sul training set e sul validation set per covare eventuali casi di overfitting, ad esempio.
\chapter{Analisi Risultati}
I test presentati sono compiuti evidenziando l'impatto scaturito della variabile momento sull'andamento dell'errore per i casi di test analizzati.\\
Le valutazioni considerano fissato il numero di epoche a $200$ e sono poste a coppie per comparare due casi per volta con valore della variabile momento rispettivamente di $0.4$ e $0.8$. Oltre al numero di epoche, sono fissati per ogni coppia il numero di neuroni e il valore del learning rate che oscilla tra $1e^{-05}$ e $1e^{-06}$. \\
Altri parametri sono: 
\begin{itemize}
    \item funzione di errore \textit{cross entropy softmax}
    \item funzione di attivazione per lo strato interno \textit{tanh}
    \item funzione per lo strato di output \textit{identità}
\end{itemize}
I parametri descritti sono presenti nel file \texttt{properties.ini}.
{\clearpage}
\section{Analisi 5 nodi interni}
\begin{center}
\includegraphics[width=0.49\linewidth]{results/5-neurons-1e-05-rate-0.4-momentum.png}
\includegraphics[width=0.49\linewidth]{results/5-neurons-1e-05-rate-0.8-momentum.png}
\captionof{figure}{Andamento errore con learning rate fissato a 1e-05, 5 neuroni con momento variabile}
\end{center}

\begin{center}
\includegraphics[width=0.49\linewidth]{results/5-neurons-1e-06-rate-0.4-momentum.png}
\includegraphics[width=0.49\linewidth]{results/5-neurons-1e-06-rate-0.8-momentum.png}
\captionof{figure}{Andamento errore con learning rate fissato a 1e-06, 5 neuroni con momento variabile}
\end{center}
\begin{table}[htbp]
    \centering
    \begin{tabular}{|c|c|c|}
    \hline
    n=5 & \multicolumn{2}{c|}{Learning rate} \\
    \hline
    Momento & 1e-05 & 1e-06 \\
    \hline
    0.4 & 0.36852 & 0.6673 \\
    \hline
    0.8 & 0.25148 & 0.49181 \\
    \hline
    \end{tabular}
    \caption{Risultati accuratezza con numero neuroni pari a 5}
\end{table}

\section{Analisi 10 nodi interni}
\begin{center}
\includegraphics[width=0.49\linewidth]{results/10-neurons-1e-05-rate-0.4-momentum.png}
\includegraphics[width=0.49\linewidth]{results/10-neurons-1e-05-rate-0.8-momentum.png}
\captionof{figure}{Andamento errore con learning rate fissato a 1e-05, 10 neuroni con momento variabile}
\end{center}

\begin{center}
\includegraphics[width=0.49\linewidth]{results/10-neurons-1e-06-rate-0.4-momentum.png}
\includegraphics[width=0.49\linewidth]{results/10-neurons-1e-06-rate-0.8-momentum.png}
\captionof{figure}{Andamento errore con learning rate fissato a 1e-06, 10 neuroni con momento variabile}
\end{center}
\begin{table}[htbp]
    \centering
    \begin{tabular}{|c|c|c|}
    \hline
    n=5 & \multicolumn{2}{c|}{Learning rate} \\
    \hline
    Momento & 1e-05 & 1e-06 \\
    \hline
    0.4 & 0.22411 & 0.17128 \\
    \hline
    0.8 & 0.21293 & 0.68277 \\
    \hline
    \end{tabular}
    \caption{Risultati accuratezza con numero neuroni pari a 10}
\end{table}

\section{Analisi 20 nodi interni}
\begin{center}
\includegraphics[width=0.49\linewidth]{results/20-neurons-1e-05-rate-0.4-momentum.png}
\includegraphics[width=0.49\linewidth]{results/20-neurons-1e-05-rate-0.8-momentum.png}
\captionof{figure}{Andamento errore con learning rate fissato a 1e-05, 20 neuroni con momento variabile}
\end{center}

\begin{center}
\includegraphics[width=0.49\linewidth]{results/20-neurons-1e-06-rate-0.4-momentum.png}
\includegraphics[width=0.49\linewidth]{results/20-neurons-1e-06-rate-0.8-momentum.png}
\captionof{figure}{Andamento errore con learning rate fissato a 1e-06, 20 neuroni con momento variabile}
\end{center}
\begin{table}[htbp]
    \centering
    \begin{tabular}{|c|c|c|}
    \hline
    n=5 & \multicolumn{2}{c|}{Learning rate} \\
    \hline
    Momento & 1e-05 & 1e-06 \\
    \hline
    0.4 & 0.23037 & 0.13401 \\
    \hline
    0.8 & 0.24481 & 0.20273 \\
    \hline
    \end{tabular}
    \caption{Risultati accuratezza con numero neuroni pari a 20}
\end{table}

\section{Analisi 30 nodi interni}
\begin{center}
\includegraphics[width=0.49\linewidth]{results/30-neurons-1e-05-rate-0.4-momentum.png}
\includegraphics[width=0.49\linewidth]{results/30-neurons-1e-05-rate-0.8-momentum.png}
\captionof{figure}{Andamento errore con learning rate fissato a 1e-05, 30 neuroni con momento variabile}
\end{center}

\begin{center}
\includegraphics[width=0.49\linewidth]{results/30-neurons-1e-06-rate-0.4-momentum.png}
\includegraphics[width=0.49\linewidth]{results/30-neurons-1e-06-rate-0.8-momentum.png}
\label{fig:n5-m0.8-l1e-05}
\captionof{figure}{Andamento errore con learning rate fissato a 1e-06, 30 neuroni con momento variabile}
\end{center}
\begin{table}[htbp]
    \centering
    \begin{tabular}{|c|c|c|}
    \hline
    n=5 & \multicolumn{2}{c|}{Learning rate} \\
    \hline
    Momento & 1e-05 & 1e-06 \\
    \hline
    0.4 & 0.21182 & 0.12214 \\
    \hline
    0.8 & 0.14753 & 0.11934 \\
    \hline
    \end{tabular}
    \caption{Risultati accuratezza con numero neuroni pari a 30}
\end{table}

\section{Analisi 50 nodi interni}
\begin{center}
\includegraphics[width=0.49\linewidth]{results/50-neurons-1e-05-rate-0.4-momentum.png}
\includegraphics[width=0.49\linewidth]{results/50-neurons-1e-05-rate-0.8-momentum.png}
\captionof{figure}{Andamento errore con learning rate fissato a 1e-05, 50 neuroni con momento variabile}
\end{center}

\begin{center}
\includegraphics[width=0.49\linewidth]{results/50-neurons-1e-06-rate-0.4-momentum.png}
\includegraphics[width=0.49\linewidth]{results/50-neurons-1e-06-rate-0.8-momentum.png}
\captionof{figure}{Andamento errore con learning rate fissato a 1e-06, 50 neuroni con momento variabile}
\end{center}
\begin{table}[htbp]
    \centering
    \begin{tabular}{|c|c|c|}
    \hline
    n=5 & \multicolumn{2}{c|}{Learning rate} \\
    \hline
    Momento & 1e-05 & 1e-06 \\
    \hline
    0.4 & 0.21775 & 0.14078 \\
    \hline
    0.8 & 0.13457 & 0.17304 \\
    \hline
    \end{tabular}
    \caption{Risultati accuratezza con numero neuroni pari a 50}
\end{table}

{\clearpage}
\section{Considerazioni}
I grafici dell'andamento dell'errore sono posti in relazione fissando il numero di neuroni e il valore del learning rate e analizzando l'andamento solo sulla variabile momento per derivarne l'impatto.\\
Si deduce dopo una prima analisi che l'andamento dell'errore sia influenzato da tale variabile, difatti il numero di epoche necessarie per scovare un \textbf{minimo locale} diminuisce; col quale la rete termina nella maggior parte delle analisi evidenziate.\\
Oltre all'influenza esercitata sulla velocità sembra anche esser presente da un certo numero di neuroni in poi, la \textbf{capacità di inasprire la curva di errore} : aumentando il valore assoluto di massimi locali e di accentuando le curve analizzate. \\
I risultati di errore ed accuratezza si dividono in due fasce :
\begin{itemize}
    \item per un numero di neuroni maggiore di 510 si ottiene un peggioramento dell'accuratezza al diminuire del learning rate. Inoltre per i casi considerati, il valore 0.8 del momento attesta quasi sempre un risultato migliore di accuratezza.
    \item al contrario sui test compiuti su 5 neuroni si nota come sia presente un massimo locale che è accentuato dall'aumentare del valore del momentum. Presente inoltre anche un drastico aumento dell'accuratezza al diminuire del learning rate che è condiviso con un numero di neuroni pari a 10, anche se in maniera meno evidente.
\end{itemize}
L'analisi evidenza quanto il valore del momento possa far accelerare l'apprendimento in regioni piatte, consentendo alla rete di apprendere con un numero minore di epoche il problema in esame.
Ricordando le ipotesi dei test effettuati indicate precedentemente (in particolare la cardinalità ridotta del training set) e derivando dalle analisi sull'accuratezza che questa cala all'aumentare dei neuroni, i test potrebbero presentare un caso di overfitting : difatti all'aumentare del numero di neuroni (simbolico è il test con 50 neuroni e learning rate $1e^{-05}$) l'errore commesso sul validation set si stabilizza mentre l'errore sul training compie miglioramenti sempre più lievi.\\
Per affrontare tale problema si potrebbe pensare di aumentare la cardinalità del training set e di applicare il criterio di \textit{early stopping}, terminando l'addestramento prima del numero stabilito di epoche se il validation set inizia la fase di stabilizzazione o di peggioramento.\\
Le valutazioni si focalizzano sull'impatto del momento sull'errore commesso, ma al contempo dimostrano anche l'impatto che determina il valore del learning rate. Il \textbf{learning rate fissa la velocità di apprendimento} di una rete e per i casi esaminati è considerato il valore $1e^{-05}$ e il valore $1e^{-06}$. Considerando tutte le coppie di test in esame con learning rate pari a $1e^{-06}$ si nota un più lento apprendimento, il che potrebbe esser equiparato da un numero maggiore di epoche. Oltre al lento apprendimento della rete, salta all'occhio anche un minore appiattimento della curva dell'errore sul validation set: tenendo conto di quanto detto prima, tale considerazione comporta un risultato relativamente migliore e che quindi con un numero maggiore di epoche sembra indicare delle performance migliori della rete rispetto ai test con valore $1e^{-05}$.Inoltre dall'analisi scaturisce quanto un valore relativamente alto per una rete possa inficiare sulle sue capacità di apprendere saltando eventuali minimi locali. \\
Riguardo l'accuratezza, i risultati migliori sono ottenuti da 5 e 10 neuroni, con un picco del $68\%$. Questo dimostra quanto aumentare solo il numero maggiore di neuroni, e di conseguenza di connessioni, possa non bastare per ottenere migliori performance ma anzi le posso peggiorare.  
\chapter{Sviluppi Futuri}
La proposta di sviluppi futuri deriva dai risultati delle analisi effettuate.\\
Si propone al lettore una serie di modifiche alla rete e alle variabili dei casi di test, tali da studiare l'andamento dell'errore e con l'itento di migliorare l'accuratezza ottenuta. \\
Proposte :
\begin{itemize}
    \item offrire la possibilità di assegnare un valore del momento e del learning rate per ogni layer
    \item aggiungere casi di test per i valori del learning rate e del momento
    \item aumentare il numero di epoche ed applicare, se ritenuto necessario, il criterio di \textit{early stopping}
    \item utilizzare altre tipologie di rete oltre alle reti feed-forward e full-connected
    \item modificare le funzioni da analizzare
    \item eseguire test su un numero di feature ridotto
    \item aumentare il numero di strati interni considerati per i test
\end{itemize}

\end{document}