 \documentclass[a4paper,11pt]{book}
\usepackage[T1]{fontenc}
\usepackage[utf8]{inputenc}
\usepackage[italian]{babel}
\usepackage{lmodern}
\usepackage{amsmath}
\usepackage{amsfonts}
\usepackage{amssymb}
\usepackage{amsthm}
\usepackage{graphicx}
\usepackage{color}
\usepackage{url}
\usepackage{theorem}
\usepackage{textcomp}
\usepackage{listings}
\usepackage{hyperref}
\usepackage{parskip}
\usepackage{soul} % for the command \hl
\usepackage{colortbl} % colore tabella
\usepackage{tikz} % per flowcharts
\usepackage{titlesec}
\usepackage{listings} % code snippet
% \usepackage[dvipsnames]{xcolor} da risolvere per ulteriori colori per snippet , come : NavyBlue
\usepackage{xcolor}
\usepackage{float} % with \begin{table}[H] for fixed table position
\usetikzlibrary{shapes, arrows} % per flowcharts
\graphicspath{{images/}}

\title{Parallel and Distribuited Computing}
\author{Schiavo Alessandro}
\date{\today}

\titleformat{\chapter}[display]
  {\normalfont\bfseries}{}{0pt}{\Huge} % remove label "Chapter" for every new chapter

% code snipper colors
\definecolor{codegreen}{rgb}{0,0.6,0}
\definecolor{codegray}{rgb}{0.5,0.5,0.5}
\definecolor{codepurple}{rgb}{0.58,0,0.82}
\definecolor{backcolour}{rgb}{0.95,0.95,0.92}

% Definig a custom style:
\lstdefinestyle{mystyle}{
    backgroundcolor=\color{backcolour},   
    commentstyle=\color{codepurple},
    %keywordstyle=\color{NavyBlue},
    keywordstyle=\color{blue},
    numberstyle=\tiny\color{codegray},
    stringstyle=\color{codepurple},
    basicstyle=\ttfamily\footnotesize\bfseries,
    breakatwhitespace=false,         
    breaklines=true,                 
    captionpos=t,                    
    keepspaces=true,                 
    numbers=left,                    
    numbersep=5pt,                  
    showspaces=false,                
    showstringspaces=false,
    showtabs=false,                  
    tabsize=2
}
% -- Setting up the custom style:
\lstset{style=mystyle}


% definizione colore headers tabelle
\definecolor{header}{RGB}{247,230,113}

\begin{document}

\begin{titlepage}
    \begin{center}
        {{\Large
        {\textsc{Università degli studi di Napoli Federico II}}}} 
        \rule[0.1cm]{15.8cm}{0.1mm}
        {\small{\bf SCUOLA POLITECNICA E DELLE SCIENZE DI BASE\\  \vspace{3mm}
        DIPARTIMENTO DI INGEGNERIA ELETTRICA E TECNOLOGIE
DELL’INFORMAZIONE \\  \vspace{3mm}
CORSO DI LAUREA MAGISTRALE IN INFORMATICA}}
    \end{center}
    \vspace{15mm}
    \begin{center}
        {\LARGE{\bf Parallel and Distributed Computing }}\\
        \vspace{3mm}
        {\LARGE{\bf Elaborato 1}}\\
    \end{center}
    \vspace{40mm}
    \par
    \noindent
    \begin{minipage}[t]{0.47\textwidth}
        {\large{\bf Professore:\\
        Giuliano Laccetti}}
    \end{minipage}
    \hfill
    \begin{minipage}[t]{0.47\textwidth}
        \raggedleft
        {\large{\bf Matricola:\\ Alessandro Schiavo}}
    \end{minipage}
    \vspace{20mm}
    \begin{center}
        {\large{\bf ANNO ACCADEMICO 2022 / 2023 }}
    \end{center}
\end{titlepage}

% \maketitle
\tableofcontents

\chapter{Definzione ed analisi del problema}
\chapter{Definzione dell'agoritmo}
\chapter{Input e Output}
\chapter{Indicatori di errore}
\chapter{Subroutine}
\chapter{Analisi dei tempi}
\chapter{Esempi d'uso}
\chapter{Appendice}

\end{document}
